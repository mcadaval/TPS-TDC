\section{Conclusiones}

Salvando el caso del experimento \textit{Cumpleaños}, las redes presentan un comportamiento parecido respecto de la proporción de mensajes de control en el tráfico de la red. Si bien no tenemos valores de referencia contra los que comparar, al tener tres redes con proporciones parecidas de mensajes \textit{broadcast} en contraste con los \textit{unicast} tenemos un buen indicio inicial de qué \textit{overhead} puede esperarse por parte de protocolos de control.

En cuando a la distinción de hosts en la red, vemos que en todos los casos nuestra definición de $S_1$ llevó a falsos negativos que, más aún, serían detectados de haber elegido las IPs de origen de los mensajes \textit{who-has}. En cambio, no hay evidencia de falsos positivos, salvo en el caso del barrido lineal en \textit{Cumpleaños}.

El criterio planteado de identificación de nodos distinguidos en base a la información y entropía en ningún caso dio el mejor resultado, que sería el de identificar únicamente al \textit{default gateway}; en todos los casos dio falsos positivos, y sólo en uno hay evidencia para suponer que lo detectó correctamente. Esto, en gran parte, se debe a comportamientos inesperados de algunos hosts. Por otra, puede atribuirse a que ocurre más seguido de lo que esperábamos que un router sea \textit{emisor} de mensajes \textit{who-has}, hecho que, como señalamos en la sección anterior, significa que hubiéramos tenido mejores resultados de identificación de \textit{default gateways} de haber elegido que los símbolos de $S_1$ sean IPs de origen.

Si bien nuestros datos no reflejan una clara relación entre la entropía y la cantidad de nodos, sí pensamos que el cálculo de la entropía y la cantidad de información provista por los hosts puede ser un dato útil a la hora de formular un criterio para clasificarlos, idea respaldada por la teoría, por la identificación parcialmente exitosa en el experimento \textit{Café} y por el haber clasificado a los routers entre los hosts de menor información en nuestros resultados.

Podemos agregar como conclusión general que, por la variedad de protocolos que existen, la naturaleza descentralizada de las redes, la pluralidad de tecnologías que se encuentran en ellas y lo heterogéneas que por lo tanto suelen resultar, desarrollar herramientas simples para estudiarlas de manera general y automática es una tarea muy difícil. La misma variabilidad de los sistemas que serían objeto de estudio de dichas herramientas hace que se requiera mínimamente también una variedad amplia de criterios para su evaluación. Como trabajo futuro, nos plantearíamos como alternativa realizar experimentos utilizando la misma herramienta presentada en este trabajo en ambientes más controlados, corriéndola sobre redes de las que se conocen características de infraestructura a un grado un poco mayor. Otra alternativa es seguir experimentando en sistemas pocos controlados para derivar criterios más precisos. Un primer candidato que tendría buen potencial en base a nuestros datos es uno que complemente el modelado de la red como una fuente $S_1$ con símbolos definidos tanto en base a direcciones de destino como de origen.
