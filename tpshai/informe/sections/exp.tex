\section{Experimentación}
Se recolectó información en distintas redes Wi-Fi durante un intervalo $\delta$t $> 10$ minutos.
\begin{itemize}
	\item Dos ambientes laborales: empresas que llamaremos \textit{ComercioLibre} y \textit{CoinFactory}.
	%\item Un caf\'e al que nos referiremos como \textit{Caf\'eCaf\'e}.
	\item Una fiesta de \textit{Cumpleaños} en un departamento.
	\item Un \textit{Café}.
\end{itemize}

\iffalse
%NO me parece IMPORTANTE, pero lo dejo por si se decide que sí (revisar la longitud del informe previo a descomentar esta parte)
\begin{center}\small
	\begin{tabular}{ l | c}
	  Red Wi-Fi & $\delta$t \\
	  \hline
	  ComercioLibre & 50 \\
      \hline
	  CoinFactory & 15 \\
      \hline
	  TiendaAntinatural & 20 \\
      \hline
	  Cumpleaños & 30 \\
	\end{tabular}
\end{center}
\fi

Con la información recolectada, cada red se modeló a través de las fuentes $S$ y $S_1$ con el objetivo de analizar los paquetes broadcast y los nodos distinguidos. Para cada fuente se calculó la información de cada símbolo en base a su frecuencia relativa y la entropía de la fuente. Naturalmente, en el caso de $S_1$ esto requiere tener registro además de la cantidad de símbolos considerados, ya que a diferencia de $S$ esa cantidad es variable.
