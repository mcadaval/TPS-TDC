\section{Introducción}
En este trabajo se plantean dos modelos de fuentes de información de memoria nula para redes de área local. Se recolecta información durante un intervalo de tiempo $\delta t$ de diferentes redes utilizando la herramienta Scapy y se concluyen relaciones entre las características físicas de las mismas y las fuentes modeladas.

\section{Definiciones}
\subsection{Fuente \textit{S}}
Sea \textit{p} un paquete capturado en el intervalo $\delta t$, se dice que \textit{p} emite un símbolo \textit{f(p)} donde
\[ \textit{f(p)} = 
     \begin{cases}
       \SimbBroadcast &\quad\text{\footnotesize si la MAC destino de \textit{p} es \textbf{FF:FF:FF:FF:FF:FF}}\\
       \SimbUnicast  &\quad\text{\footnotesize de otro modo}\\
     \end{cases}
\]
De esta forma \textit{S} es una fuente binaria con un conjunto de símbolos $\{ \SimbBroadcast, \SimbUnicast\}$. Notar que, de esta manera, la entropía máxima de $S$ (i.e su entropía bajo equiprobabilidad) es $log_2(2) = 1$.

Esta definición parte de la noción de que, generalmente, las comunicaciones en una red que llevan información de nivel de aplicación entre hosts suelen ser \textit{unicast}, como sería el caso por ejemplo de un router que envía a un host específico el contenido de nivel de aplicación proveniente de internet. En cambio, los protocolos de control en las redes de área local suelen implicar la realización comunicaciones \textit{broadcast} en la misma: en \textit{ARP}, por ejemplo, uno de los dos mensajes en un intercambio típico y exitoso es un \textit{broadcast}, es decir, el 50\%; en \textit{DHCP}, la proporción análoga es del 25\%. De esta manera, estudiar el comportamiento de una red modelada como una fuente $S$ puede dar una pauta de qué carga tienen los protocolos de control sobre el tráfico de una red donde se espera que predomine el tráfico generado por aplicaciones.

\subsection{Fuente \textit{S}$_1$}
Sea \textit{p} un paquete \textit{ARP} \textit{who-has} capturado en el intervalo $\delta t$, decimos que \textit{p} emite un símbolo
\begin{center}
	\textit{g(p)} = \textit{IP} destino de \textit{p}
\end{center}

\subsection{Identificación de hosts en base a $S_1$}
La definición de la fuente tiene como objetivo que, al modelar una red como una fuente $S_1$, permita identificar hosts (\textit{nodos}) en esa red y establecer un criterio para clasificar algunos de ellos como distinguidos.

\medskip

\textbf{Símbolo ditinguido} Sean $\{ s_1,\dots, s_n \}$ el conjunto de símbolos de una fuente \textit{F}. Decimos que $s_i$ es un símbolo distinguido de \textit{F} si
\begin{equation}
	\textit{I}(s_i) < \textit{H(F)}
\end{equation}
donde \textit{I} es la información y \textit{H} la entropía, notando así que $s_i$ brinda menos información que el promedio o, dicho de otra forma, la probabilidad de recibir $s_i$ es mayor al resto.

\medskip

Notemos que los paquetes ARP \textit{who-has} se envían cuando un host desea conocer la dirección física de otro host. Esto puede ser desencadenado por cualquier intento de comunicaciones a nivel red entre hosts, por ejemplo cuando un router desea enviarle información proveniente de internet a alguno en particular o cuando un host desea enviarle información al router para que la dirija a internet. Es por eso que, al elegir que los símbolos de $S_1$ sean IPs de destino de tales mensajes, se espera poder reconocer a los hosts de cada red. En este sentido, notemos que los nodos distinguidos son aquellos cuya dirección física se encuentra entre las más solicitadas; por ende, podemos pensar que entre ellos se encontrarían los routers de cada red monitoreada.

Es importante observar que a priori un host en una red puede preguntar por la dirección física de cualquier IP, esté en la red o no, lo cual llevaría a que $S_1$ emita falsos positivos en su función de identificación de nodos. Por ello, al elegir que los símbolos de $S_1$ sean las IPs de destino de mensajes \textit{who-has}, estamos suponiendo que eso no ocurre en gran medida. Consideramos que, mínimamente, tiene sentido suponer que no se enviarán tales mensajes a IPs que ni siquiera podrían pertenecer a la red, hipótesis respaldada en la teoría por el hecho de poder conocer los hosts su máscara de subred.

Al mismo tiempo, si se da que un host conectado a la red nunca es destino de una consulta \textit{who-has}, pasaría desapercibido por nuestra captura y tendremos casos de falsos negativos. Esto puede ocurrir por distintos motivos, como en particular el hecho de que, según la bibliografía\footnote{\peterson} y el RFC del protocolo \footnote{\rfcDeArp}, el receptor de un \textit{who-has} actualiza su conocimiento respecto de la información del emisor y puede prevenir así algunos de los \textit{who-has} que de lo contrario podría enviarle en momentos consiguientes. Sin embargo, por lo explicado anteriormente, no es nada esperable que \textit{un router} termine constituyendo un falso negativo.